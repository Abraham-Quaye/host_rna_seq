% Options for packages loaded elsewhere
\PassOptionsToPackage{unicode}{hyperref}
\PassOptionsToPackage{hyphens}{url}
%
\documentclass[
]{article}
\usepackage{amsmath,amssymb}
\usepackage{iftex}
\ifPDFTeX
  \usepackage[T1]{fontenc}
  \usepackage[utf8]{inputenc}
  \usepackage{textcomp} % provide euro and other symbols
\else % if luatex or xetex
  \usepackage{unicode-math} % this also loads fontspec
  \defaultfontfeatures{Scale=MatchLowercase}
  \defaultfontfeatures[\rmfamily]{Ligatures=TeX,Scale=1}
\fi
\usepackage{lmodern}
\ifPDFTeX\else
  % xetex/luatex font selection
\fi
% Use upquote if available, for straight quotes in verbatim environments
\IfFileExists{upquote.sty}{\usepackage{upquote}}{}
\IfFileExists{microtype.sty}{% use microtype if available
  \usepackage[]{microtype}
  \UseMicrotypeSet[protrusion]{basicmath} % disable protrusion for tt fonts
}{}
\makeatletter
\@ifundefined{KOMAClassName}{% if non-KOMA class
  \IfFileExists{parskip.sty}{%
    \usepackage{parskip}
  }{% else
    \setlength{\parindent}{0pt}
    \setlength{\parskip}{6pt plus 2pt minus 1pt}}
}{% if KOMA class
  \KOMAoptions{parskip=half}}
\makeatother
\usepackage{xcolor}
\usepackage[margin=1in]{geometry}
\usepackage{graphicx}
\makeatletter
\def\maxwidth{\ifdim\Gin@nat@width>\linewidth\linewidth\else\Gin@nat@width\fi}
\def\maxheight{\ifdim\Gin@nat@height>\textheight\textheight\else\Gin@nat@height\fi}
\makeatother
% Scale images if necessary, so that they will not overflow the page
% margins by default, and it is still possible to overwrite the defaults
% using explicit options in \includegraphics[width, height, ...]{}
\setkeys{Gin}{width=\maxwidth,height=\maxheight,keepaspectratio}
% Set default figure placement to htbp
\makeatletter
\def\fps@figure{htbp}
\makeatother
\setlength{\emergencystretch}{3em} % prevent overfull lines
\providecommand{\tightlist}{%
  \setlength{\itemsep}{0pt}\setlength{\parskip}{0pt}}
\setcounter{secnumdepth}{-\maxdimen} % remove section numbering
% definitions for citeproc citations
\NewDocumentCommand\citeproctext{}{}
\NewDocumentCommand\citeproc{mm}{%
  \begingroup\def\citeproctext{#2}\cite{#1}\endgroup}
\makeatletter
 % allow citations to break across lines
 \let\@cite@ofmt\@firstofone
 % avoid brackets around text for \cite:
 \def\@biblabel#1{}
 \def\@cite#1#2{{#1\if@tempswa , #2\fi}}
\makeatother
\newlength{\cslhangindent}
\setlength{\cslhangindent}{1.5em}
\newlength{\csllabelwidth}
\setlength{\csllabelwidth}{3em}
\newenvironment{CSLReferences}[2] % #1 hanging-indent, #2 entry-spacing
 {\begin{list}{}{%
  \setlength{\itemindent}{0pt}
  \setlength{\leftmargin}{0pt}
  \setlength{\parsep}{0pt}
  % turn on hanging indent if param 1 is 1
  \ifodd #1
   \setlength{\leftmargin}{\cslhangindent}
   \setlength{\itemindent}{-1\cslhangindent}
  \fi
  % set entry spacing
  \setlength{\itemsep}{#2\baselineskip}}}
 {\end{list}}
\usepackage{calc}
\newcommand{\CSLBlock}[1]{\hfill\break\parbox[t]{\linewidth}{\strut\ignorespaces#1\strut}}
\newcommand{\CSLLeftMargin}[1]{\parbox[t]{\csllabelwidth}{\strut#1\strut}}
\newcommand{\CSLRightInline}[1]{\parbox[t]{\linewidth - \csllabelwidth}{\strut#1\strut}}
\newcommand{\CSLIndent}[1]{\hspace{\cslhangindent}#1}
\usepackage{helvet}
\renewcommand*\familydefault{\sfdefault}
\usepackage{setspace}
\doublespacing
\usepackage[left]{lineno}
\linenumbers
\ifLuaTeX
  \usepackage{selnolig}  % disable illegal ligatures
\fi
\usepackage{bookmark}
\IfFileExists{xurl.sty}{\usepackage{xurl}}{} % add URL line breaks if available
\urlstyle{same}
\hypersetup{
  pdftitle={Transcriptomics of a THEV-infected Turkey B-cell Line},
  hidelinks,
  pdfcreator={LaTeX via pandoc}}

\title{Transcriptomics of a THEV-infected Turkey B-cell Line}
\author{}
\date{\vspace{-2.5em}}

\begin{document}
\maketitle

\vspace{5mm}

Abraham Quaye\({^\dagger}\)\textsuperscript{,a}, Brian D.
Poole\textsuperscript{a,*}

\vspace{5mm}

\textsuperscript{a}Department of Microbiology and Molecular Biology,
Brigham Young University\\
\({^\dagger}\)First-author\\
\textsuperscript{*}Corresponding Author

\vspace{5mm}

\textbf{Corresponding Author Information}\\
\href{mailto:brian_poole@byu.edu}{\nolinkurl{brian\_poole@byu.edu}}\\
Department of Microbiology and Molecular Biology,\\
4007 Life Sciences Building (LSB),\\
Brigham Young University,\\
Provo, Utah\\

\newpage

\subsection{ABSTRACT}\label{abstract}

\newpage

\subsection{INTRODUCTION}\label{introduction}

Turkey hemorrhagic enteritis virus (THEV), a virus isolated from
turkeys, chickens, and pheasants, belongs to the family
\emph{Adenoviridae}, genus \emph{Siadenovirus} (1, 2). Infecting its
hosts via the feco-oral route, THEV causes hemorrhagic enteritis (HE) in
turkeys, a debilitating disease affecting predominantly 6-12 week old
poults characterized by immunosuppression (IMS), splenomegaly,
intestinal lesions leading to bloody diarrhea, and up to 80\% mortality
(3--6). The clinical disease usually persists in affected flocks for
about 7--10 days. However, secondary bacterial infections may extend the
duration of illness and mortality for an additional 2--3 weeks due to
the immunosuppressive nature of the virus, exacerbating the economic
losses (5, 7). Low pathogenic (avirulent) strains of THEV have been
isolated, which show subclinical infections but retain their
immunosuppressive effects. One such avirulent strain called Virginia
Avirulent Strain (VAS) is used as a live vaccine; thus, vaccinated birds
are rendered more susceptible to opportunistic infections and death than
unvaccinated birds leading to significant economic losses (4, 5, 8, 9).

It is well-established that THEV primarily infects and replicates in
turkey B-cells and macrophages of the bursa and spleen, inducing
apoptosis and necrosis. Consequently, a significant drop in number of
B-cells (specifically, IgM+ B-cells) and macrophages ensue along with
increased T-cell counts with abnormal T-cell subpopulation ratios. The
cell death seen in the B-cells and macrophages is generally proposed as
the cause of THEV-induced IMS as both humoral and cell-mediated immunity
are impaired (5, 6, 8, 10). It is also thought that a humoral immune
response may contribute to the IMS as follows. The virus replication in
the spleen attracts T-cells and peripheral blood macrophages to the
spleen where T-cells are activated by cytokines from infected
macrophages and vice versa. The activated T-cells proliferate and
secrete interferons: type I (IFN-\(\alpha\) and IFN-\(\beta\)) and type
II (IFN-\(\gamma\)) as well as tumor necrosis factor (TNF) while
activated macrophages secrete interleukin 6 (IL-6), TNF, and nitric
oxide (NO), an antiviral with immunosuppressive properties. The
inflammatory cytokines released by T cells and macrophages (e.g., TNF
and IL-6) may also induce apoptosis in bystander splenocytes,
exacerbating the already numerous apoptotic and necrotic splenocytes,
culminating in IMS (8, 10) (see \textbf{Figure 1}). However, the precise
molecular mechanisms and pathways of THEV-induced IMS has not been
studied.

\begin{itemize}
\tightlist
\item
  Discuss NGS here
\end{itemize}

To eliminate the immunosupressive effect of the vaccine strain, it is
essential to elucidate the host mechanisms/pathways influenced by the
virus to bring about IMS. Elucidating the mechanisms of THEV-induced IMS
is the most crucial step in THEV research as it will present a means of
mitigating IMS.

\begin{itemize}
\tightlist
\item
  Discuss the hemorrhagic enteritis disease
\item
  Discuss proposed mechanisms/pathways/ideas
\item
  Discuss why NGS will help elucidate the host response and show
  examples of NGS used to as such
\item
  End with the study aims/goals
\end{itemize}

Introduction: RNA-Seq and Differential Gene Expression: Briefly
introduce RNA sequencing (RNA-seq) as a powerful tool for studying gene
expression. Explain how RNA-seq can identify differentially expressed
genes between infected and uninfected cells. Highlight that this
approach allows us to explore potential pathways affected by THEV.
Objectives of Your Study: Clearly state your research objectives:
Identify differentially expressed genes in MDTC-RP19 turkey B-cells
infected with THEV. Investigate pathways associated with
immunosuppression caused by THEV. Methods: Describe how you obtained
RNA-seq data from infected and uninfected cells. Mention any
preprocessing steps (quality control, normalization, etc.). Briefly
outline the statistical analysis for identifying differentially
expressed genes. Expected Outcomes: Anticipate that you'll discover
specific genes upregulated or downregulated in infected cells. Expect to
find pathways related to immune response modulation affected by THEV.
Significance and Implications: Discuss the importance of understanding
THEV-induced immunosuppression. Highlight potential applications, such
as developing targeted therapies or improving turkey health management.

\url{https://link.springer.com/article/10.1007/s11259-014-9596-z}
\url{https://bioone.org/journals/avian-diseases/volume-61/issue-1/11506-092916-Reg/Molecular-Characterization-of-Hemorrhagic-Enteritis-Viruses-HEV-Detected-in-HEV/10.1637/11506-092916-Reg.full}
\newpage

\subsection{RESULTS}\label{results}

\newpage

\subsection{DISCUSSION}\label{discussion}

\newpage

\subsection{CONCLUSIONS}\label{conclusions}

\newpage

\subsection{MATERIALS AND METHODS}\label{materials-and-methods}

\textbf{Cell culture and THEV Infection}

\textbf{RNA extraction and Sequencing}

\textbf{Quality Control and Mapping Process}

\textbf{Functional Enrichment Analysis}

\textbf{Expression Profiling and Differentially Expressed Genes}

\textbf{Quantitative Real-Time Reverse Transcriptase PCR}

\textbf{Statistical Analysis} \newpage

\subsection{DATA AVAILABILITY}\label{data-availability}

\newpage

\subsection{CODE AVAILABILITY}\label{code-availability}

\newpage

\subsection{ACKNOWLEDGMENTS}\label{acknowledgments}

\newpage

\subsection{REFERENCES}\label{references}

\newpage

\subsection{TABLES AND FIGURES}\label{tables-and-figures}

\newpage

\subsection*{SUPPLEMENTARY
INFORMATION/MATERIALS}\label{supplementary-informationmaterials}
\addcontentsline{toc}{subsection}{SUPPLEMENTARY INFORMATION/MATERIALS}

\phantomsection\label{refs}
\begin{CSLReferences}{0}{1}
\bibitem[\citeproctext]{ref-Harrach2008}
\CSLLeftMargin{1. }%
\CSLRightInline{Harrach B. 2008.
\href{https://doi.org/10.1016/B978-012374410-4.00680-4}{Adenoviruses:
General features}, p. 1--9. \emph{In} Mahy, BWJ, Van Regenmortel, MHV
(eds.), Encyclopedia of virology (third edition). Book Section. Academic
Press, Oxford.}

\bibitem[\citeproctext]{ref-Davison2003}
\CSLLeftMargin{2. }%
\CSLRightInline{Davison A, Benko M, Harrach B. 2003.
\href{https://doi.org/10.1099/vir.0.19497-0}{Genetic content and
evolution of adenoviruses}. The Journal of general virology
84:2895--908.}

\bibitem[\citeproctext]{ref-Gross1967}
\CSLLeftMargin{3. }%
\CSLRightInline{Gross WB, Moore WE. 1967. Hemorrhagic enteritis of
turkeys. Avian Dis 11:296--307.}

\bibitem[\citeproctext]{ref-Beach2006}
\CSLLeftMargin{4. }%
\CSLRightInline{Beach NM. 2006.
\href{http://scholar.lib.vt.edu/theses/available/etd-08142006-145339/}{Characterization
of avirulent turkey hemorrhagic enteritis virus: A study of the
molecular basis for variation in virulence and the occurrence of
persistent infection}. Thesis.}

\bibitem[\citeproctext]{ref-Dhama2017}
\CSLLeftMargin{5. }%
\CSLRightInline{Dhama K, Gowthaman V, Karthik K, Tiwari R, Sachan S,
Kumar MA, Palanivelu M, Malik YS, Singh RK, Munir M. 2017.
\href{https://doi.org/10.1080/01652176.2016.1277281}{Haemorrhagic
enteritis of turkeys -- current knowledge}. Veterinary Quarterly
37:31--42.}

\bibitem[\citeproctext]{ref-Tykaowski2019}
\CSLLeftMargin{6. }%
\CSLRightInline{Tykałowski B, Śmiałek M, Koncicki A, Ognik K, Zduńczyk
Z, Jankowski J. 2019.
\href{https://doi.org/10.1186/s12917-019-2138-8}{The immune response of
young turkeys to haemorrhagic enteritis virus infection at different
levels and sources of methionine in the diet}. BMC Veterinary Research
15.}

\bibitem[\citeproctext]{ref-Pierson2008}
\CSLLeftMargin{7. }%
\CSLRightInline{Pierson F, Fitzgerald S. 2008. Hemorrhagic enteritis and
related infections. Diseases of Poultry 276--286.}

\bibitem[\citeproctext]{ref-Rautenschlein2000}
\CSLLeftMargin{8. }%
\CSLRightInline{Rautenschlein S, Sharma JM. 2000.
\href{https://doi.org/10.1016/s0145-305x(99)00075-0}{Immunopathogenesis
of haemorrhagic enteritis virus (HEV) in turkeys}. Dev Comp Immunol
24:237--46.}

\bibitem[\citeproctext]{ref-Larsen1985}
\CSLLeftMargin{9. }%
\CSLRightInline{Larsen CT, Domermuth CH, Sponenberg DP, Gross WB. 1985.
Colibacillosis of turkeys exacerbated by hemorrhagic enteritis virus.
Laboratory studies. Avian Dis 29:729--32.}

\bibitem[\citeproctext]{ref-Rautenschlein2000b}
\CSLLeftMargin{10. }%
\CSLRightInline{Rautenschlein S, Suresh M, Sharma JM. 2000.
\href{https://doi.org/10.1007/s007050070083}{Pathogenic avian adenovirus
type II induces apoptosis in turkey spleen cells}. Archives of Virology
145:1671--1683.}

\end{CSLReferences}

\end{document}
